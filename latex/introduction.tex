\specialsection{Введение}

С каждым днем цифровое пространство все больше интегрируется в нашу повседневную жизнь. В современной экономике наблюдается рост влияния мошенничества, обусловленный быстрым развитием финансовых технологий, таких как необанкинг, и широким распространением социальных сетей и мессенджеров.

Согласно оценкам Центрального банка России \cite{cbr2024}, только за 2023-й год общий объем операций без согласия клиентов достиг рекордных 15.8 млрд. рублей, что на 11.48\% превысило статистику предшествующего года. Более половины таких транзакций совершается с применением методов фишинга и социальной инженерии, которые активно эволюционируют в условиях развития интернета.

По данным исследования компании BI.ZONE \cite{bizone2023}, мошенничество в интернете остается одним из наиболее значимых вызовов для защиты в цифровом пространстве. За последний год число фишинговых ресурсов выросло на 86\%, причем злоумышленники чаще всего создают поддельные сайты, имитирующие популярные онлайн-площадки.

Этот негативный тренд требует непрерывного совершенствования текущих и разработки новых методов выявления аномалий и подозрительных паттернов поведения участников сети. В связи с этим исследование методов обнаружения мошеннической активности становится не только актуальным, но и стратегически важным для обеспечения цифровой безопасности.

Выбор нейронных сетей обусловлен их уникальной способностью адаптироваться к сложным и динамичным структурам данных. Это делает их эффективным инструментом анализа в условиях постоянно меняющихся природы и характера цифрового мошенничества. Применение таких подходов уже оказывает существенное влияние на различные сферы, включая, но не ограничиваясь:

\begin{itemize}
    \item Электронная коммерция: улучшение алгоритмов обнаружения мошенничества в банковских операциях и онлайн-торговле, ускорение выявления махинаций с платежами.

    \item Кибербезопасность: разработка инновационных методов выявления аномалий в сетевом трафике и предотвращение цифровой преступности с акцентом на оперативное реагирование на новые сценарии атак.

    \item Онлайн-медиа: использование нейронных сетей для нахождения ложных новостей и манипуляций с открытыми источниками данных, повышение их достоверности.
\end{itemize}

Данная работа посвящена теме определения мошеннических отзывов на сайтах-агрегаторах рецензий на товары и услуги, таких как Amazon и Yelp. Применяя решения, исследованные ниже, можно обезопасить клиентов от ошибочных покупок и укрепить доверие к платформам, что имеет важное значение для повышения общей безопасности в цифровом пространстве и защиты прав потребителей.

\pagebreak
