\specialsection{Выводы}

В результате проведенного исследования были выполнены следующие задачи:

\begin{itemize}
    \item Рассмотрена проблема и определены необходимые для оценки метрики в классических терминах области машинного обучения.

    \item Проанализированы актуальные подходы, применяемые к поставленной задаче и основанные на графовом представлении исходных данных.

    \item Оценены результаты различных архитектур на доступных и широко распространенных наборах примеров мошеннических отзывов.
\end{itemize}

Таким образом, все поставленные почти задачи были выполнены, а цель достигнута. В ходе работы не удалось только оценить и адаптировать под задачу архитектуру Exphormer \cite{shirzad2023}, что было вызвано ограничениями используемых ресурсов.

Кроме того, полученные результаты демонстрируют практическую применимость рассмотренных моделей в условиях реальных данных на примере сайтов-агрегаторов рецензий Amazon и Yelp. Это подтверждает их потенциал для использования в системах модерации пользовательских отзывов. Однако исследование было ограничено доступными вычислительными ресурсами, что повлияло на количество протестированных моделей и объем данных.

\pagebreak



\specialsection{Заключение}

В данной работе были изучены и проанализированы актуальные подходы к определению мошеннических отзывов и самих недоброжелателей на крупнейших сайтах-агрегаторах Amazon и Yelp соответственно. Была сформулирована математическая постановка задачи, описаны главные метрики качества и воспроизведены результаты нескольких наилучших моделей на языке программирования Python. На основе проделанной работы можно сделать следующие выводы:

\begin{itemize}
    \item Модель GAGA, разделяющая соседей вершины по их классам перед агрегацией, показала наилучшие метрики качества в обеих задачах при доступности открытого исходного кода.

    \item Архитектура NGS, также работающая с гетерогенными графами, продемонстрировала схожие с GAGA результаты, хотя их воспроизведение оказалось невозможным.

    \item Базовый подход агрегации (пространственная парадигма), представленный в GraphSAGE, показал значительное преимущество в качестве на рассматриваемых задачах по сравнению с модулями GAT и GCN.
\end{itemize}

Для дальнейшего улучшения результатов в данной области можно рассмотреть следующие направления:

\begin{enumerate}
    \item \textbf{Оптимизация моделей}: Все исследуемые выше работы предъявляют значительные требования к ресурсам системы, на которой они запускаются. Можно изучить зависимость итогового качества от выбранных параметров обучения, чтобы добиться тех же результатов при меньших ресурсных затратах.

    \item \textbf{Совмещение подходов}: К рассматриваемой графовой структуре можно добавить методы обработки естественного языка для извлечения дополнительных признаков из текстов рецензий.

    \item \textbf{Масштабирование архитектур}: Улучшение основных модулей, как, например, это делают авторы Exphormer \cite{shirzad2023}, может способствовать достижению больших масштабов и точности моделей.
\end{enumerate}

Таким образом, результаты данной работы подтверждают перспективность использования графовых нейронных сетей для определения мошенничества и предоставляют базу для дальнейших исследований и разработок в этой области.

Рассмотренные модели могут быть интегрированы в системы модерации отзывов крупных платформ, таких как Amazon и Yelp, для автоматического выявления мошеннических рецензий и самих недоброжелателей.

\pagebreak
